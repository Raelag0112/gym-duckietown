\chapter*{Introduction}

Reinforcement Rearning (RL) is an area of machine learning based on trial and error.
The goal is for an agent to learn an optimal behaviour and adapt its comportement, based on experience, in a given environment.
Reinforcement Learning has been very successful in the last few years, notably for its great performance in games (e.g. AlphaGo \cite{rlgames}).
However, it suffers from a lack of industrial applications.

One of the reason for this is that an agent needs a lot of experience in order to learn a reasonable behaviour.
Those learning experiences must be run in a simulator for safety, speed and financial reasons.
The transfer of behaviour from the simulated agent to the real world (for example an autonomous vehicle) is far from being trivial.
Even for high quality simulators, there is a shift in both states and transitions spaces between the simulator and the real world.

The goal of this project is to study different methods to robustly train an autonomous car in a simulator via RL, in order for the real-world agent to behave as suited.

All the project experiments will be based on Duckietown\footnote{\url{https://www.duckietown.org/}} resources.
The Duckietown fundation is a non-profit foundation providing tools dedicated to education and research in the fields of AI and robotics.
In particular, it provides a gym-based simulator (gym-duckietown), a robot (Duckiebot) and a real world environment (Duckietown).

This document aims at providing an understanding of the basic concepts required to start working on Duckietown environments.

% TODO Developement plan
% TODO Bibtex github

A fork of the gym-duckietown repository\cite{gym_duckietown} is dedicated to this project\cite{forked_gym_duckietown}.
The latest version of this manual can be found on the branch documentation.
