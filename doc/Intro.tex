\chapter*{Introduction}

Reinforcement Rearning is an area of machine learning. 
The goal is for an agent to learn an optimal behaviour and adapt its comportement, based on experience, in a given environment. 
Reinforcement Learning has been very successful in the last few years, notably for its great performance in games (e.g. AlphaGo\footnote{Silver, D., Schrittwieser, J., Simonyan, K. et al. Mastering the game of Go without human knowledge. Nature 550, 354–359 (2017)}). 
However, it suffers from a lack of industrial applications.

One of the reason for this is that an agent needs a lot of experience in order to learn a reasonable behaviour. 
Those simulations must be run in a simulator for safety, speed and financial reasons. 
Moreover, the transfer of behaviour from the simulated agent to the real world (for example an autonomous vehicle) is far from being trivial.
Even for high quality simulators, there is a shift in both states and transitions spaces between the simulator and the real world.

The purpose of this project is to study different methods to robustly train an autonomous car in a simulator, so that the real-world agent behaves as suited.


DuckieTown\footnote{See https://www.duckietown.org/} environment will be used during all the project.
It provides a simulator (Gym-Duckietown), a real-world agent and a real-world environment. 

