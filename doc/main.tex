% A LaTeX (non-official) template for ISAE projects reports
% Copyright (C) 2014 Damien Roque
% Version: 0.2
% Author: Damien Roque <damien.roque_AT_isae.fr>

\documentclass[a4paper,12pt,twoside]{report}
\usepackage[utf8]{inputenc}
\usepackage[T1]{fontenc}
\usepackage[english]{babel}
\usepackage{roboto}
\usepackage{a4wide}
\usepackage{graphicx}
\graphicspath{images/}
\usepackage{subfig}
\usepackage{tikz}
\usetikzlibrary{shapes,arrows}
\usepackage{pgfplots}
\pgfplotsset{compat=newest}
\pgfplotsset{plot coordinates/math parser=false}
\newlength\figureheight
\newlength\figurewidth
\pgfkeys{/pgf/number format/.cd,
set decimal separator={,\!},
1000 sep={\,},
}
\usepackage{ifthen}
\usepackage{ifpdf}
\ifpdf
\usepackage[pdftex]{hyperref}
\else
\usepackage{hyperref}
\fi
\usepackage{color}
\hypersetup{%
colorlinks=true,
linkcolor=black,
citecolor=black,
urlcolor=black}

\renewcommand{\baselinestretch}{1.05}
\usepackage{fancyhdr}
\pagestyle{fancy}
\fancyfoot{}
\fancyhead[LE,RO]{\bfseries\thepage}
\fancyhead[RE]{\bfseries\nouppercase{\leftmark}}
\fancyhead[LO]{\bfseries\nouppercase{\rightmark}}
\setlength{\headheight}{15pt}

\let\headruleORIG\headrule
\renewcommand{\headrule}{\color{black} \headruleORIG}
\renewcommand{\headrulewidth}{1.0pt}
\usepackage{colortbl}
\arrayrulecolor{black}

\fancypagestyle{plain}{
  \fancyhead{}
  \fancyfoot[C]{\thepage}
  \renewcommand{\headrulewidth}{0pt}
}

\makeatletter
\def\@textbottom{\vskip \z@ \@plus 1pt}
\let\@texttop\relax
\makeatother

\makeatletter
\def\cleardoublepage{\clearpage\if@twoside \ifodd\c@page\else%
  \hbox{}%
  \thispagestyle{empty}%
  \newpage%
  \if@twocolumn\hbox{}\newpage\fi\fi\fi}
\makeatother

\usepackage{amsthm}
\usepackage{amssymb,amsmath,bbm}
\usepackage{array}
\usepackage{bm}
\usepackage{multirow}
\usepackage[footnote]{acronym}

\newcommand*{\SET}[1]  {\ensuremath{\mathbf{#1}}}
\newcommand*{\VEC}[1]  {\ensuremath{\boldsymbol{#1}}}
\newcommand*{\FAM}[1]  {\ensuremath{\boldsymbol{#1}}}
\newcommand*{\MAT}[1]  {\ensuremath{\boldsymbol{#1}}}
\newcommand*{\OP}[1]  {\ensuremath{\mathrm{#1}}}
\newcommand*{\NORM}[1]  {\ensuremath{\left\|#1\right\|}}
\newcommand*{\DPR}[2]  {\ensuremath{\left \langle #1,#2 \right \rangle}}
\newcommand*{\calbf}[1]  {\ensuremath{\boldsymbol{\mathcal{#1}}}}
\newcommand*{\shift}[1]  {\ensuremath{\boldsymbol{#1}}}

\newcommand{\eqdef}{\stackrel{\mathrm{def}}{=}}
\newcommand{\argmax}{\operatornamewithlimits{argmax}}
\newcommand{\argmin}{\operatornamewithlimits{argmin}}
\newcommand{\ud}{\, \mathrm{d}}
\newcommand{\vect}{\text{Vect}}
\newcommand{\sinc}{\ensuremath{\mathrm{sinc}}}
\newcommand{\esp}{\ensuremath{\mathbb{E}}}
\newcommand{\hilbert}{\ensuremath{\mathcal{H}}}
\newcommand{\fourier}{\ensuremath{\mathcal{F}}}
\newcommand{\sgn}{\text{sgn}}
\newcommand{\intTT}{\int_{-T}^{T}}
\newcommand{\intT}{\int_{-\frac{T}{2}}^{\frac{T}{2}}}
\newcommand{\intinf}{\int_{-\infty}^{+\infty}}
\newcommand{\Sh}{\ensuremath{\boldsymbol{S}}}
\newcommand{\C}{\SET{C}}
\newcommand{\R}{\SET{R}}
\newcommand{\Z}{\SET{Z}}
\newcommand{\N}{\SET{N}}
\newcommand{\K}{\SET{K}}
\newcommand{\reel}{\mathcal{R}}
\newcommand{\imag}{\mathcal{I}}
\newcommand{\cmnr}{c_{m,n}^\reel}
\newcommand{\cmni}{c_{m,n}^\imag}
\newcommand{\cnr}{c_{n}^\reel}
\newcommand{\cni}{c_{n}^\imag}
\newcommand{\tproto}{g}
\newcommand{\rproto}{\check{g}}
\newcommand{\LR}{\mathcal{L}_2(\SET{R})}
\newcommand{\LZ}{\ell_2(\SET{Z})}
\newcommand{\LZI}[1]{\ell_2(\SET{#1})}
\newcommand{\LZZ}{\ell_2(\SET{Z}^2)}
\newcommand{\diag}{\operatorname{diag}}
\newcommand{\noise}{z}
\newcommand{\Noise}{Z}
\newcommand{\filtnoise}{\zeta}
\newcommand{\tp}{g}
\newcommand{\rp}{\check{g}}
\newcommand{\TP}{G}
\newcommand{\RP}{\check{G}}
\newcommand{\dmin}{d_{\mathrm{min}}}
\newcommand{\Dmin}{D_{\mathrm{min}}}
\newcommand{\Image}{\ensuremath{\text{Im}}}
\newcommand{\Span}{\ensuremath{\text{Span}}}

\usepackage[style=numeric]{biblatex}
\usepackage{csquotes}
\addbibresource{articles.bib}

\newtheoremstyle{break}
  {11pt}{11pt}%
  {\itshape}{}%
  {\bfseries}{}%
  {\newline}{}%
\theoremstyle{break}

%\theoremstyle{definition}
\newtheorem{definition}{Définition}[chapter]

%\theoremstyle{definition}
\newtheorem{theoreme}{Théorème}[chapter]

%\theoremstyle{remark}
\newtheorem{remarque}{Remarque}[chapter]

%\theoremstyle{plain}
\newtheorem{propriete}{Propriété}[chapter]
\newtheorem{exemple}{Exemple}[chapter]

\parskip=5pt
%\sloppy


\usepackage{graphicx} % This lets you include figures
\usepackage{hyperref} % This lets you make links to web locations
\graphicspath{ {./images/} }

\usepackage[rightcaption]{sidecap}
\usepackage{wrapfig}

\usepackage{float}

\usepackage{imakeidx}

\usepackage{listings}

\usepackage{pgf-umlsd}

\usepackage{amssymb}

\makeindex




\begin{document}

%%%%%%%%%%%%%%%%%%
%%% First page %%%
%%%%%%%%%%%%%%%%%%

\begin{titlepage}
\begin{center}

\includegraphics[width=0.65\textwidth]{IRT.jpg}\\[1cm]



% Title
\rule{\linewidth}{0.5mm} \\[0.4cm]
{ \huge \bfseries Project 074 : Simulation to Real\\[0.4cm] }

{\large Study on Gym-Duckietown environment, supervised by David Bertoin }\\[0.5cm]

\rule{\linewidth}{0.5mm} \\[1.5cm]

% Author and supervisor
\noindent
\begin{minipage}{0.29\textwidth}
    \begin{center} \large
    \emph{Auteurs :}\\
        Olivier SUTTER\\
        Mathieu VERM\\
        Michael ROMAGNE\\
        Alexandre MARTIN\\
        Pablo MIRALLES\\
        Vincent COYETTE\\
    \end{center}
\end{minipage}%

\smallbreak
\vfill


\smallbreak
\includegraphics[width=0.26\textwidth]{logo-isae-supaero}\\[1cm]

% Bottom of the page
{\large \today}
\end{center}

\end{titlepage}


\tableofcontents

\clearpage
\newpage


\chapter*{Introduction}

Reinforcement Rearning (RL) is an area of machine learning.
The goal is for an agent to learn an optimal behaviour and adapt its comportement, based on experience, in a given environment.
Reinforcement Learning has been very successful in the last few years, notably for its great performance in games (e.g. AlphaGo \cite{rlgames}).
However, it suffers from a lack of industrial applications.

One of the reason for this is that an agent needs a lot of experience in order to learn a reasonable behaviour.
Those learning experiences must be run in a simulator for safety, speed and financial reasons.
Moreover, the transfer of behaviour from the simulated agent to the real world (for example an autonomous vehicle) is far from being trivial.
Even for high quality simulators, there is a shift in both states and transitions spaces between the simulator and the real world.

The purpose of this project is to study different methods to robustly train an autonomous car in a simulator via RL, in order for the real-world agent to behave as suited.

All the project experiments will be based on Duckietown\footnote{\url{https://www.duckietown.org/}} resources.
The Duckietown fundation is a non-profit foundation providing tools dedicated to education and research in the fields of AI and robotics.
In particular, it provides a gym-based simulator (gym-duckietown), a robot (Duckiebot) and a real world environment (Duckietown).

This document aims at providing an understanding of the basic concepts required to work on the project.

% TODO Developement plan

A fork of the gym-duckietown repository is dedicated to this project\footnote{\url{https://github.com/vcoyette/gym-duckietown}}.
The latest version of this manual can be found on the branch documentation.



\chapter{Duckietown environment}

A reinforcement learning agent is usually trained in an environment described as a Markov Decision Process (MDP).
A MDP is basically a 4-tuple $(S, A, P_a, R_a)$.
Let's detail each of these components for the duckietown environment.

\section{State}
The state S is the state of the environment from the agent point of view.
Here, the robot contains only one sensor, the camera.
The state of this environment is the output of this camera.
The camera has a resolution of 640x480, and thus the state space is $S = [0, 255]^{640x480x3}$ (3 is for the 3 RGB colors).

\section{Actions}
The action is composed by the velocities of each of the two wheels.
By default, the action space is continuous, $A = [-1, 1]^2$, $1$ being full forward velocity, $-1$ full backward.

\textbf{Caution :}
Be careful here as the definition of the actions in the README of the duckietown repository is describing actions differently.
It also defines a two-element tuple, but the first one represents the forward velocity while the second one represents the steering angle.
This definition is more natural to manually control the robot.
A wrapper (DuckietownEnv) exists to switch from the control of wheels velocity to the control of forward velocity and streering.
However, we chose not to use this wrapper and to keep the default implementation.
The code contains some inconsistencies without the wrapper, which will be described later.

\section{Transitions}
The transitions of a MDP is the distribution $P(s'| s, a)$, i.e. the probability of ending up in the state $s'$ if the agent is in the state s and take the action a.
This is typically a point where the transfer from the simulator to the robot may be harmful.

Indeed, the transitions in the simulator can be considered as "perfect", the position of the robot in state $s'$ is computed from the position in state $s$ using the velocities stated by action $a$.
This is not fully deterministic, as some objects (such as duckie pedestrians) may have a stochastic behaviour, but the updated position is a deterministic function of the previous position and current action.

However, in the real world, the transitions between the positions are not perfect.
Given a position and an action, the next position is not deterministic.
It can slightly vary because of mechanical links not being perfect, because of unusual road surface...
This shift in the transition distribution is one of the problem we will have to address during the simulation to reality transfer.

\section{Reward}
The choice of the reward function is primordial to properly train the agent. The reward may depend on :

\begin{itemize}
    \item The position of the agent. The position may be used to compute the distance from a target position, and the distance from the middle of the lane.
    \item The speed of the agent. The faster the better.
    \item Collisions. The agent may get a negative reward in case of collision, and possibly if it gets too  close to another object.
\end{itemize}

The default reward in duckietown is as follows:

\begin{align}
R(t) = 40 * C_{p} - 10*dist + alignment*speed \\
C_p = \sum_{Obj} ((pos_{agent} - pos_{object}) - SR_{agent} - SR_{object}) \\
\end{align}
\begin{itemize}
    \item $C_p$ is the collision penalty for being dangerously close to other objects. It is a proxy for area overlap. Notably, one can collide with several objects at once (with additive effects) and one can collide with respect to the reward function (which uses the safety radius) without the episode restarting (which depends on the collision radius). Note that collision penalty is smaller than 0 whenever there is a collision, therefore this is actually a penalty despite the positive sign.
    \item dist is the distance between the agent's center and the closest point on the line defining the lane's center
    \item alignment is the dot product between direction and the normalized tangent of the road.
    \item speed is the agent's speed
\end{itemize}

This reward function has presented several issues:

\begin{enumerate}
    \item The penalties (collision and distance) are so strong that the agent usually much prefers to go around in circles at max speed at the center of the lane than to actually follow it.
\end{enumerate}

To face this issues, a minimalist reward function has been used :
\begin{equation}
    R =
    \begin{cases}
        speed, & \text{if}\ dist \leq d\\
        -1, & \text{otherwise}
    \end{cases}
\end{equation}
Where $d$ is a constant defining a minimal distance to the center of the line, over which we consider the robot should be penalised. We used a value $d=0.1\ m$.

The design of the reward function will condition the performance of our agent in the simulator.
This reward will be used to train a policy, i.e. a mapping from a state $s$ to an action $a$ to take.
This policy will need to be adapted before transfer to the real world.
However, the reward does not need to be adapted.

\chapter{Gym-DuckieTown Simulator}

From the README.md in the official github repository \cite{gym_duckietown}, here is an introduction to Gym-Duckietown.

 \begin{quotation}
    Gym-Duckietown is a simulator for the Duckietown Universe, written in pure Python/OpenGL (Pyglet). It puts the agent, a Duckiebot, inside of an instance of a Duckietown: a loop of roads with turns, intersections, obstacles, Duckie pedestrians, and other Duckiebots. It can be a pretty hectic place!
\end{quotation}

 This chapter provides a guide through installation, usage and architecture of the simulator.

\section{Installation}

The installation is pretty straight-forward from the source code. Use the following commands :

\begin{lstlisting}[language=bash]
    git clone https://github.com/vcoyette/gym-duckietown
    cd gym-duckietown
    conda env create -f environment.yaml
\end{lstlisting}

The installation has been tested on Windows, Linux and MacOS. Some problems may be encountered for the installation of certain packages.
They can be resolved with package-specific installation instructions.
For example, the installation of pyglet may raise an issue.
It can be resolved by installing it from the pyglet github repository.

To use the simulator, the environment must be activated (on linux) :

\begin{lstlisting}[language=bash]
    source activate gym-duckietown
\end{lstlisting}

And the root folder of the project must be added to the PYTHONPATH environment variable.

On linux :
\begin{lstlisting}[language=bash]
    export PYTHONPATH="${PYTHONPATH}:`pwd`"
\end{lstlisting}

On Windows, environment variable can be accessed in windows advanced parameters. You can then append the path of your project folder to the PYTHONPATH variable if it exists, or create it otherwise.

\section{Manual Control}
A UI application can be launched to manually control the robot.
Actions can be sent from the keyboard arrows, and images from the simulated DuckieBot camera are displayed.
Here is a simple command to launch the application :

\begin{lstlisting}{laguage=bash}
    ./manual_control.py --env-name Duckietown-udem1-v0
\end{lstlisting}

The map can be specified through the \lstinline[language=bash]+"--map-name"+ environment.

\subsection{Keyboard}
Here is a list of keys which can be used during the simulation.

\begin{itemize}
	\item Escape : exit simulation
	\item Backspace : restart simulation
	\item Directional keys: go forward, backward or turn
	\item Shift : boost speed
\end{itemize}

\subsection{Logs}
The project repository\cite{forked_gym_duckietown} contains a branch "experiment".
This branch is supposed to be used to do any experiment on the simulator.
The main difference from a classical "develop" branch is that it does not aim at being merged into the master branch.

This branch contains a logger for the manual control application.
At each time step, this logger will log the current position of the agent, the current speed, the current distance from the lane, the action took and the reward value.
The goal is to manually control the agent so that it behaves as expected.
The logs can then be used to improve the design of the reward function for example.

To enable this option, pass the \lstinline[language=bash]+--output+ option to \path{manual\_control.py}.
You can optionally specify another option \lstinline[language=bash]+--filename example.csv+ to specify the name of the output file, which would be \path{data/example.csv}.
If no file-name is specified, the logs will be stored in \path{data/manual\_controli.csv}, where $i$ is the first number for which this path is free.

When the episode is done, the manual control must be exited by pressing the S key to save the output.


\section{Create Maps}
You can very easily create a new \textbf{Duckietown} environment with a text editor.
A Duckietown's map is a \textit{.yaml} file, so you have to save your new map in the folder \path{maps} as "\path{my_new_map.yaml}" (the path should looks like this one: \path{./gym-duckietown-master/gym_duckietown/maps}).

If you want to see your map, you can use the following line in the terminal:

\begin{lstlisting}{language=bash}
    ./manual_control.py --env-name Duckietown-udem1-v0
    --map-name my_new_map
\end{lstlisting}

\noindent Your robot will be manually controlled in your map.

\noindent A grassroots level of your \textit{.yaml} should looks like this :

\begin{center}
    \line(1,0){400}
\end{center}
\noindent tiles:

\noindent- [floor, floor, floor, grass, grass, grass, floor, floor]\newline
- [floor, floor, grass, grass, straight/S, grass, floor, floor]\newline
- [floor, grass, grass, curve\_left/W, curve\_right/S, grass, floor, floor]\newline
- [grass, grass, curve\_left/W, curve\_right/S, grass, grass, floor, floor]\newline
- [grass, curve\_left/W, curve\_right/S, grass, grass, floor, floor, floor]\newline
- [grass, curve\_left/S, curve\_right/E, grass, grass, floor, floor, floor]\newline
- [grass, grass, curve\_left/S, curve\_right/E, grass, floor, floor, floor]\newline
- [floor, grass, curve\_left/W ,curve\_right/S, grass, floor, floor, floor]\newline
- [floor, grass, straight/S, grass, grass, floor, floor, floor]\newline
- [floor, grass, grass, grass, floor, floor, floor, floor]\newline
- [floor, floor, floor, floor, floor, floor, floor, floor]

\noindent objects:

\noindent- kind: house\newline
  pos: [4.5, 9.1]\newline
  rotate: 90\newline
  height: 0.5

\noindent- kind: tree\newline
  pos: [1, 1]\newline
  rotate: 0\newline
  height: 0.5

\noindent- kind: tree\newline
  pos: [2, 8.5]\newline
  rotate: 90\newline
  height: 0.3

\noindent tile\_size: 0.585
\begin{center}
    \line(1,0){400}
\end{center}

For each line, the number of tiles has to remain the same.
The \textbf{tile\_size} and the \textbf{height} of every object can change, but insofar as your goal is to transpose the simulation to the real world, you must not change their values.
If so, you should have \textbf{tile\_size}$=0.585$ (for the conventional heights of the objects will be given below).

You can find your way in the map knowing that going upward is going North, and knowing that when you drive on a road tile, the name of the road tile tells you which direction you were facing when you arrived on it, and by which direction you will leave the tile.
Let's give an example : the tile "\textit{curve\_left/W}" means that you arrived on it while you were moving West, and the fact that you'll turn left on it says you are going to leave the tile by the South (it's of course reversible, you can arrive from South facing North, and leave to the East).
You can also know your position $(x,\ y)\in\mathbb{R}^2$ on the map (this is the way you put objects on it).
The point $[0,\ 0]$ matches the upper left corner of the upper left tile of the map (so the coordinates start at the very North/West).
The coordinates keep going higher as you move toward South/East, increasing of 1 each time you go through one tile.

Here is the list of the tiles you can use to build your map:
\begin{itemize}
    \item straight
    \item curve\_left
    \item curve\_right
    \item 3way\_left (3-way intersection)
    \item 3way\_right
    \item 4way (4-way intersection)
    \item asphalt
    \item grass
    \item floor (office floor)
\end{itemize}

You can (and should) orientate the roads (the six first tiles on the list) by adding "/N", "/E", "/S", "/W" (this will be oriented according to the rule given above).

The objects can be added as shown in the following example (changing the name of the object). Here is a list of them :

\begin{itemize}
    \item barrier (height : 0.08)
    \item cone (height : 0.08)
    \item duckie (height : as you wish between 0.06 and 0.08)
    \item duckiebot (height : 0.12)
    \item tree (height : as you wish between 0.1 and 0.9)
    \item house (height : 0.5)
    \item truck (height : 0.25)
    \item bus (height : 0.18)
    \item building (height : 0.6)
    \item sign\_stop, sign\_T\_intersect, sign\_yield, etc... (height : $0.18$ for the signs, $0.4$ for a traffic light)
\end{itemize}

There are many other signs, you can check the whole list here :
\url{https://github.com/duckietown/gym-duckietown/blob/master/gym\_duckietown/meshes}

It is possible to add the attributes :

\begin{itemize}
    \item optional: True or False (makes the object optional)
    \item static: True or False (for the Duckiebot for example if you want to see them move)
\end{itemize}

To go any further about map creation, check the Github of Duckietown on this link : \footnote{$https://github.com/vcoyette/gym-duckietown/tree/documentation$ }

\section{Domain randomization}

When it comes to transfer knowledge from simulation to reality, a problem which may be face is the images collected from simulation diverging too much from reality.
Often, people even retrain their model from scratch when moving to the physical world.

One solution to this problem is domain randomization.
The idea is to perturb the dynamics or look of the simulator such as colors, textures, horizon...
This achieves a more variable dataset and better generalization, which is beneficial for transfer to the real robot.
This section will deal with domain randomization in gym-duckietown environments.

\subsection{The randomisation API}

The folder \path{gym_duckietown/randomization} contains the domain randomization API.
This API contains all of the pre-packaged methods for randomization within gym-duckietown, which will be listed here.
This folder also contains a readme file detailing the API.

The domain randomization is driven by the Randomizer class, which takes as input a configuration file and outputs (upon call to randomize) a set of settings used by the Simulator class (the core class of gym\-duckietown managing the environment).
To activate any domain randomization at all, the simulator class must have domain\_rand = true passed as a parameter to its constructor.

If a randomizable variable is not found in the configuration file, it will be randomized according to the default values found in \url{gym-duckietown/gym_duckietown/randomization/config/default.json}.
Three types of distribution are supported, int, uniform and normal. 4 variables are randomizable by default:
\begin{enumerate}
    \item horz\_mode: The task is made harder by making the horizon more similar to the road. It can take integer values from 0 to 3 where:
        \begin{itemize}
            \item 0: Sets the skybox to a blue sky, the default
            \item 1: Sets the skybox to a gray wall, intended to be similar to room testing conditions
            \item 2: Sets the skybox to a dark gray box
            \item 3: Sets the skybox to a light gray box
        \end{itemize}
    \item light\_pos: Makes the simulator more or less iluminated, by changing the position of the single light source.
    \item camera\_noise: Adds noise to the camera position for data augmentation purposes. This noise is applied at each render, giving the screen input a "twitchy" behaviour.
    \item frame\_skip: No info provided. Code inspection shows this is the number of frames to skip per action. Higher frameskip makes the agent able to act less frequently.
\end{enumerate}

The API is sorely lacking in the amount of variables that can be randomized.
It provides some flexibility in randomization of the input, but it provides no randomization besides frame\_skip of transitions and/or actions.

\subsection{Non API Randomisation}

Code inspection reveals that more randomisation occurs, contingent on domain-rand being activated.
\begin{enumerate}
    \item Within Simulator.py: All calls to \_perturb(val,scale) distort the value passed as argument if and only if domain rand is true, otherwise they return it undisturbed. The distortion is a multiplicative \% error drawn uniformly between $1 - scale$ and $1 + scale$, with a default of $scale = 0.1$ (e.g. by default values are randomised between 90 and 110\%). The function is called on:
        \begin{itemize}
            \item horizon\_color, beyond the randomization introduced by horz\_mode. 10\% for blue\_sky and wall\_color sky (modes 1 and 2), 40\% for modes 3 (dark gray) and 4 (clear gray)
            \item glLight function : sets the values of individual light source parameters, making the environment more or less iluminated. It modifies the light position, the ambient and diffusion of light.
            \item ground\_color, 30\%
            \item wheel\_dist, 10\%
            \item cam\_height, 8\%
            \item cam\_angle, 20\%
            \item cam\_fov\_y, 20\%, camera field of view
                side length of the ground/noise triangles generated as distractors (which themselves are generated randomly in the first place? Seems redundant to do it twice), 10\%
            \item tile color, 20\% for each tile
            \item object color, 20\% for each color
            \item CAMERA\_FORWARD\_DIST of gl.glTranslatef on line 1434, 10\%
            \item The actual tile texture loaded is randomised with randint amongst all possible candidates
            \item Some optional objects are invisible, 33\% chance
        \end{itemize}
    \item Within objects.py
        \begin{itemize}
            \item DuckiebotObj (Cars)
                \begin{itemize}
                    \item follow\_dist is randomised from 0.3 to a uniform between 0.3 and 0.4
                    \item velocity is randomised from 0.1 to a uniform between 0.05 and 0.15
                \end{itemize}
            \item DuckieObj (Pedestrians)
                \begin{itemize}
                    \item pedestrian\_wait\_time randomised from 8 to a randint from 3 to 20, takes on new random value on same range when finish crossing street
                    \item vel randomised from 0.02 to a normal with avg 0.02 and stdev 0.005, takes on new random value on same range when finish crossing street
                \end{itemize}
            \item TrafficLightObj
                \begin{itemize}
                    \item freq is randomised from 5 with randint(4,7)
                    \item The lights start randomly from off as either On or Off, 50\% chance each
                \end{itemize}
        \end{itemize}
\end{enumerate}


\subsection{Randomizing inputs}


Randomizing inputs can be accomplished via the API. Other possible sources of randomization are:
\begin{itemize}
    \item Changing hue of key elements: stop signs, road… Might be hard since they’re all texture based. A possible approach: Generating variations on existing textures: Automatically generate X diff textures with some other software from existing textures passed through some filters, then use the in-built texture randomiser
    \item Adding noise to the camera acquisition (not position)? This is similar to randomising the color of the sky and objects themselves so it might not be interesting. However camera noise provides variations during a single episode or each timestep which may imitate real camera noise (or not?)
    \item Approaching photorrealism in some way can be helpful in improving the performance (Johnson-Robertson et al.) May prove unfeasible given the simulator and available time and resources
    \item Addding additional light sources?
    \item Gaussian noise foreground of images and edge blurring through gaussian noise at edges on input images is also helpful (Hinterstoisser et al.) The technique can be extended to other blending techniques (Dwibedi et al.)
\end{itemize}


\chapter{Reinforcement Learning Training}

The environment implements the gym API, especially the reset, step and render functions which makes it easy to implement RL algorithms, or to use existing libraries.

The branch master of the repository\cite{forked_gym_duckietown} contains the original reinforcement learning implementation from Duckietown, while the branch develop contains our tweaked implementation.
This chapter provides an overview of the original implementation as well as details of the choices that have been made to increase learning performances.

\section{Training Scipt}

The learning is performed using the \path{train_reinforcement.py} script in the \path{learning/reiforcement/pytorch} directory.
It should be executed as a module from the \textit{learning} folder:

\begin{lstlisting}{laguage=bash}
    cd learning
    python -m reinforcement.pytorch.train_reinforcement
\end{lstlisting}

If an error is returned stating that gym-duckietown module doesn't exist, check if the PYTHONPATH environment variable contains the project root folder.

\begin{lstlisting}{laguage=bash}
    echo $PYTHONPATH
\end{lstlisting}

If it doesn't:
\begin{enumerate}
    \item On Linux/Mac OS:
        Get back to the root directory of the project and add it to the PYTHONPATH:
        \begin{lstlisting}{laguage=bash}
            cd ../
            export PYTHONPATH="${PYTHONPATH}:`pwd`"
        \end{lstlisting}
    \item On Windows:
% TODO add windows instructions

\end{enumerate}

The training can be tweaked by passing some hyperparameters when executing the train\_reinforcement module.
A list of this parameters can be accessed by running:

\begin{lstlisting}{language=bash}
    python -m reinforcement.pytorch.train_reinforcement --h
\end{lstlisting}

\section{Training Algorithm}

The original training algorithm implemented by duckietown is ddpg\cite{ddpg}.
To improve learning speed and stability, a td3\cite{td3} algorithm has been implemented.

The algorithm can be specified through the \lstinline[language=bash]+--policy+ argument, accepting as a string the name of the algorithm to use to train the agent.
Currently available values are ddpg and td3, ddpg being the default.

The policies are defined in their own files (e.g. \path{ddpg.py} and \path{td3.py}).

\section{Prioritize Experience Replay}%

A Prioritize Experience Replay\cite{per} has been implemented.
It can be used in training script using the \lstinline[language=bash]+--per+ parameter.
It is for now effective with the ddpg algorithm, \textbf{not with the td3}.

\section{Wrappers}
To improve the performances of the learning, wrappers can be used to tweak the environment without editing its code.
Wrappers implements the Adapter/Wrapper pattern\footnote{\url{https://en.wikipedia.org/wiki/Adapter_pattern}}.
They are subclasses of gym wrapper classes RewardWrapper, ObservationWrapper and ActionWrapper.
They are instanciated specifying the environment to be wrapped as a constructor parameter.

Let's detail the reward wrapper implementation as an example to review wrappers principle.
The reward wrapper intercepts the result of the step function, and replace the value of reward with a custom reward function.
Figure \ref{fig:rewardwrappers} is a sequence diagram showing how a RewardWrapper interacts with a client class.

\begin{figure}
    \begin{sequencediagram}
        \newthread{t}{Client}
        \newinst[1]{w}{RewardWrapper}
        \newinst[2]{s}{Simulator}
        \begin{call}{t}{step()}{w}{\shortstack{
                    return obs, \\ reward(reward) \\
            done, \\ info }}


            \begin{call}{w}{step()}{s}{\shortstack{
                        return obs, \\ reward, \\
                done, \\ info }}
                \postlevel
                \postlevel
                \postlevel
            \end{call}
            \postlevel
            \postlevel
            \postlevel
        \end{call}
    \end{sequencediagram}

    \caption{RewardWrapper Sequence Diagram}
    \label{fig:rewardwrappers}
\end{figure}

ObservationWrapper can be used by overriding the observation(observation) function and ActionWrapper by overriding the action(action) function.
Wrappers contain the original environment as attribute, and can thus access any information from the context.

This section presents the wrappers that have been used for the training of this project.
Wrappers are defined in the \path{learning/utils/wrappers.py} class.

\subsection{Observation Wrappers}
The orginal state space is $S = \{ 0, 255 \}^{640x480x3}$.
This is a very heavy state, especially if we use neural networks to predict the actions from the state.
Thus, the 640x480 pixels are resized to 80x60 pixels and are converted to grayscale images.
This transforms the state space from $S = \{ 0, 255 \}^{640x480x3}$ to $S = \{ 0, 255 \}^{80x60}$.

In order for the robot to have access to informations about its speed and acceleration, 4 observations are stacked into a state.
Thus, the state space is now $S = \{ 0, 255 \}^{80x60x4}$.

Finally, the observations are normalized to [0, 1], resulting in a $[0, 1]^{80x60x4} space$.

\subsection{Reward Wrapper}

% TODO add ref to reward chapter 1
As stated in the first chapter, the original reward function was not really effective during tests.

To try to improve performances, a minimalist reward function has been tested:
\begin{equation}
    R =
    \begin{cases}
        speed, & \text{if}\ dist \leq d\\
        -1, & \text{otherwise}
    \end{cases}
\end{equation}
Where $d$ is a constant defining a minimal distance to the center of the line, over which we consider the robot should be penalised. We used a value $d=0.1\ m$.

The implementation is decoupled from the architecture of actors and critics network, which are defined in \path{actor.py} and \path{critic.py} in the folder \path{learning/reinforcement/pytorch/}.

\section{Actor and Critic}%

@Oliv
Tu peux citer \cite{actorcritic} comme reference.
% TODO actor critic

\chapter{Transfer to the Duckiebot}

\section{Assembly and operation of the duckiebot}

For the most part we recommend following the duckiebot operating manual at \url{https://docs.duckietown.org/DT19/opmanual\_duckiebot/out/index.html}.
The guide provides a step by step process on how to succesfully assemble, boot, and operate the duckiebot.

While the guide is helpful, Section D does not provide help on how to troubleshoot many of the issues that can appear during boot. In this section we provide information on how to deal with the problems we faced However we recommend heading to the duckietown slack community over at \url{duckietown.slack.com}, which as is very active as of March 2020 and proved invaluable to us. Their considerable expertise and willingness to help can aid you in troubleshooting your issues.

\subsection{Assembly of the duckiebot}

Note: We strongly recommend flashing the SD card and performing the first boot before mounting the Raspberry Pi on the duckiebot. Connect a monitor and/or a keyboard to the raspberry pi to aid you in this procedure, otherwise your sole source of information on the boot process will be the red and green LED lights, until you are able to connect the bot to the local network and ssh into it. After mounting the raspberry PI, the HDMI and keyboard input ports will be rendered inaccessible.

\subsubsection{Inserting and extracting the SD card after mounting the Raspberry PI}

Mounting will render the SD card port of the Raspberry PI hard to access. We recommend using a set of calipers to access the SD card and insert or extract it.


\subsection{Flashing the SD Card and booting the Raspberry Pi}

daffy is the experimental version. It contains numerous duckietown shell commands available for it exclusivel, that can simplify troubleshooting significantly. dts-19 is the last stable build, although it is comparatively lacking in capabilities.

Here are some of the issues that we've encountered upon setting up the duckiebot and how to solve them

\subsubsection{Booting the Raspberry PI before assembly}

The first boot process of raspberry PI is often faulty. As mentioned at the beginning of the section, to ease troubleshooting, we recommend performing this process BEFORE mounting the Raspberry PI onto the duckiebot to make the procedure easier to follow
\subsubsection{Interpreting the LED lights}
The duckiebot operating manual mentions that the following pattern is to be observed during a successful boot of the duckiebot:

\begin{enumerate}
    \item Stable red LED light and intermittent green LED light
    \item Alternating red and green LED lights
    \item Persistent green LED light - The process succeeded
    \item A persistent red LED light means the process failed somehow
\end{enumerate}

The guide does not explain properly that this process is ONLY to happen during the FIRST boot of the raspberry PI. Subsequent boots will have the red LED light lit permanently whereas the green LED light just marks activity on the SD port. Do not be alarmed by this.

\subsubsection{Failed boot procedure - containers missing}

Very often (100\% of the time so far), the boot procedure will fail silently - several of the required packages will fail to be installed with no apparent error message during the flashing procedure or on the duckiebot's health logs. Should that be the case, we recommend that you try to download these containers individually from docker and set them up yourself. The specifics of this process vary case by case. The only alternative is to reflash the SD card and reboot the Raspberry Pi, a time-consuming procedure with low success probabilities.

ATTACH IMAGE OF ALL CONTAINERS NEEDED FOR REGULAR OPERATIONS

\subsection{Loading an agent}

TO BE COMPLETED

\subsection{Calibrating the camera}

TO BE COMPLETED

\subsection{Calibrating the wheels}

TO BE COMPLETED

\end{document}


\printbibliography

\end{document}



