%\documentclass{article}
%\usepackage[utf8]{inputenc}

\documentclass[12pt]{article}
\usepackage{graphicx} % This lets you include figures
\usepackage{hyperref} % This lets you make links to web locations
\graphicspath{ {./images/} }

\usepackage[rightcaption]{sidecap}
\usepackage{subcaption}
\usepackage{wrapfig}

\usepackage{float}

\usepackage{imakeidx}

\usepackage{listings}

\usepackage{pgf-umlsd}


\makeindex


\title{SIM2REAL}
\author{Olivier Sutter \and Mathieu Verm \and Michael Romagne \and Alexandre Martin \and Pablo Mirales \and Vincent Coyette}

\date{\today}

\begin{document}
\maketitle{}

\tableofcontents

\clearpage
\newpage

\section{Introduction}
Reinforcement learing is an area of machine learning. The goal is for an agent to learn an optimal behaviour to adopt in a given environment based on experience. Reinforcement Learning has been very successful in the last few years, notably for its great performance in games (e.g. AlphaGo\footnote{Silver, D., Schrittwieser, J., Simonyan, K. et al. Mastering the game of Go without human knowledge. Nature 550, 354–359 (2017)}). However, it suffers from a lack of industrial applications.

One of the reason for this is the fact that an agent needs a lot of experiences in order to learn an acceptable behaviour. Those simulations must be run in a simulator for security, speed and economic reasons. The transfer of behaviour from the simulated agent to the real world (for example an autonomous vehicle) is far from being trivial. Indeed, even for high quality simulators, there is a shift in both spaces of states and transitions between the simulator and the real world.

The goal of this project is to study different methods to robustly train an autonomous car in a simulator, in order for the real-world agent to behave as suited.


DuckieTown\footnote{See https://www.duckietown.org/} environment will be used during all the project. It provides both a simulator (Gym-Duckietown) and real-world agent and environment. 


\section{Environment}
An environment in which a Reinforcement Learning agent is trained is modelled as an Markov Decision Process (MDP). An MDP is a 4-tuple $(S, A, P_a, R_a)$. Let's detail each of these component for duckietown. 

\subsection{State}
In an MDP, the state is defined as the state of the environment as it is seen by the agent. Here, the robot contains only one captor, the camera. The camera has a resolution of 160x120, and thus the state space is $S = [0, 255]^{160x120x3}$ (3 is for the 3 RGB colors).

\subsection{Actions}
The actions the environment can take are composed by a forward velocity and a steering angle. By default, the action space is continuous, $A = [-1, 1]^2$:
\begin{itemize}
    \item The first number correspond to the forward velocity. 1 is for going full speed forward, -1 is for full speed backward.
    \item The second number correspond to the steering angle. 1 is for the steering wheel being fully to the left.
\end{itemize}

The action space can be changed to be discrete. In this case, the possible actions would be move forward, turn left or turn rigth.

\subsection{Transitions}
The transition of a MDP is the distribution $P(s'| s, a)$, i.e. the probability of ending up in the state $s'$ if the agent is in the state s and take the action a. The transitions would typically be different inside the simulator and in the real life.

Inside the simulator, the transitions are considered as perfect. The agent being in an initial position and the taken action being ${speed, angle}$, the simulator will update the position of the agent accordingly. The state s' is then the new camera signal. The state is not fully deterministic, as some objects (such as duckie pedestrians for example, may have a stocastic behaviour and appear on the camera), but the updated position of the agent is a deterministic function of the current
position and the current action.

In the real world however, the transitions of positions are not perfect. Given a position and an action, the next position is not deterministic. It can slightly vary because of mechanical links not being perfect, because of unusual road surface.. This shift in the transition distribution is one of the problem we will have to address during the simulation to reality transfer.

\subsection{Reward}
The choice of the reward function is primordial to correctly train the agent. The reward may depend on :

\begin{itemize}
    \item The position of the agent. The position of the agent on the driving line is important, a position in the middle of a line must be prefered. The position may also be used to compute the distance to a target position.
    \item The speed of the agent. The agent may be rewarded if it is going straight forward. 
    \item Collisions. An agent must have a negative reward in case of collision, and possibly if it gets to close to another object.
\end{itemize}

The design of the reward function will condition the performance of our agent in the simulator. This reward will be use to train a policy, i.e. a mapping from a state s to an action a to take. This policy will need to be adapted before transfer to real world. However, the reward may not need to be adapted. 

\section{Gym-DuckieTown Simulator}

From the README.md in the github repository\footnote{https://github.com/duckietown/gym-duckietown}, here is an introduction to Gym-Duckietown. 

 \begin{quotation}
  Gym-Duckietown is a simulator for the Duckietown Universe, written in pure Python/OpenGL (Pyglet). It places your agent, a Duckiebot, inside of an instance of a Duckietown: a loop of roads with turns, intersections, obstacles, Duckie pedestrians, and other Duckiebots. It can be a pretty hectic place!
 \end{quotation}

 This project has a github dedicated repository\footnote{https://github.com/vcoyette/gym-duckietown}, which is a fork from the Gym-Duckietown original repo.

\subsection{Installation}
The installation is pretty straight-forward from the repository. Use the following commands :

\begin{lstlisting}[language=bash]
    git clone https://github.com/vcoyette/gym-duckietown
    cd gym-duckietown
    conda env create -f environment.yaml
\end{lstlisting}

The installation have been tested on Windows, Linux and MacOS. Some problems may be encountered for the installation of certain packages. They can be resolved with package-specific installation instructions. 
For example, the installation of pyglet may raise an issue. It can be resolved by installing it from source from the pyglet github repository. 

To use the simulator, the environment must be activated :
 
\begin{lstlisting}[language=bash]
    conda activate gym-duckietown
\end{lstlisting}

And the root folder of the project must be add to the PYTHONPATH environment variable.
On linux :
\begin{lstlisting}[language=bash]
    export PYTHONPATH="${PYTHONPATH}:`pwd`"
\end{lstlisting}

On Windows, environment variable can be accessed in the parameters, section advanced parameters. You can then append the path to your project folder to the PYTHONPATH variable exists, or create it otherwise.


\subsection{Manual Control}
A UI application can be launched to manually control the robot. Actions can be sent from the keyboard, and images from the DuckieBot camera are diplayed. Here is a simple command to launch the application :

\begin{lstlisting}{laguage=bash}
$ ./manual_control.py --env-name Duckietown-udem1-v0
\end{lstlisting}

\subsubsection{Parameters}
Here is a list of parameters which can be used. 

\begin{itemize}
	\item env-name: the name of the environment to execute (TODO anchor to env description)
	\item map-name: the name of the map to be used
	\item distortion: boolean, add distorsion
	\item draw-curve: boolean, draw the lane-following curve
	\item draw-bbox: boolean, draw the collision bounding box
	\item domain-rand: boolean, use domain randomization
	\item frame-skip: number of frames to skip (default 1)
	\item seed: seed
\end{itemize}

\subsubsection{Keyboard}
Here is a list of keys which can be used during the simulation.

\begin{itemize}
	\item Escape : exit simulation
	\item Backspace : restart simulation
	\item Directional keys: go forward, backward or turn
	\item Shift : boost speed
\end{itemize}

\subsubsection{Logs}
The github repository contains a branch "experiment". This branch is intended to be used to do any experiment on the simulator. The difference to a classical "develop" branch is that it is not aimed to be merged into the master branch. 

In this branch, we designed a logger for the manual control application. This logger will at each time step log the current position of the agent, the current speed, the current distance to the line, the action took. The objective is to manually control the agent to behave as we would expect it to behave. We would then be able to check the logs, and use them to design a reward function.

To enable this option, pass the --output option to manual\_control.py. You can optionally specify another option `--filename example.csv` to specify the name of the output file, which would be in data/example.csv. If no file-name is specified, the logs will be stored in data/manual\_controli.csv, where i is the first number for which this path is free.

When the episode is done, the manual control must be exit by pressing the S key to save the output.

\subsection{Learning}
The learning is performed using the train\_reinforcement.py script. In order to execute it, change into the directory learning and run : 

\begin{lstlisting}{laguage=bash}
    python -m reinforcement.pytorch.train_reinforcement
\end{lstlisting}

If an error is returned stating that a module doesn't exist, check that PYTHONPATH variable contains the repository base folder.

\begin{lstlisting}{laguage=bash}
    echo $PYTHONPATH
\end{lstlisting}

If it doesn't, get back to your to the root directory of the project and run: 

\begin{lstlisting}{laguage=bash}
    export PYTHONPATH="${PYTHONPATH}:`pwd`"
\end{lstlisting}

The default algorithm implemented is DDPG. It can be tweaked by passing some parameters when executing train\_reinforcement. A list of the parameters can be accessed by running :

\begin{lstlisting}{language=bash}
    python -m reinforcement.pytorch.train_reinforcement --h
\end{lstlisting}

\subsection{Reward Wrapper}
The reward function is computed by the compute\_reward function defined in the class Simulator (in the file gym\_duckietown/simulator.py). The peferred way of modifying the reward function is to edit the function reward of the DtRewardWrapper class in the file learning/utils/wrappers.py.

This class is a subclass of the gym RewardWrapper class. This class implements the Wrapper/Adapter design pattern. It overrides the step function of the simulator. It is hard to explain, but you can have a look at the following sequence diagram to get an idea of how it works.

The idea is that the reward wrapper replaces the environment in the DDPG implementation. During the training, the DDPG calls the step function of the wrapper. This wrapper instance has an attribute which is the environment. The step
function in the wrapper calls the function step of the environment, and captures the returned values (observation, reward, done and info). Then, the wrapper runs its own reward function with as parameters the reward returned by the previous call. The wrapper step function returns the values of observation, done and info which were returned by the environment, and returns its custom reward to replace the reward of the environment.

The wrapper containing a environment attribute, it has access to all the same information as the simulator to compute its custom reward. For example, it can use the current postion through env.cur\_pos, the angle through env.angle, etc.


\begin{sequencediagram}
    \newthread{t}{DDPG}
    \newinst[1]{w}{DtRewardWrapper}
    \newinst[2]{s}{Simulator}
    \begin{call}{t}{step()}{w}{\shortstack{
        return obs, \\ reward(reward) \\
        done, \\ info }}
    

        \begin{call}{w}{step()}{s}{\shortstack{
            return obs, \\ reward, \\
            done, \\ info }}
            \postlevel
            \postlevel
            \postlevel
        \end{call}
    \postlevel
    \postlevel
    \postlevel
    \end{call}
\end{sequencediagram}

TODO : add legend to diagram






\end{document}


